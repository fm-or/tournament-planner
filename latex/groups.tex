\documentclass[
	headinclude=false,
	footinclude=false,
	]{scrartcl}
    
\usepackage[a4paper,margin=20mm]{geometry}

\usepackage[T1]{fontenc}% moderne Schriftkodierung
\usepackage[utf8]{inputenc}% Sonderzeichen vieler verschiedener Sprachen gleichzeitig (€, ° usw.)
\usepackage[english,main=ngerman]{babel}% Primär Deutsch und sekundär Englisch
\usepackage[autostyle]{csquotes}% Anführungszeichen mit \enquote{} passen zur Sprache von babel
\usepackage{lmodern}% Schriftfamilie latin modern basierend auf computer modern verträgt sich besser mit fontenc T1
\usepackage{microtype}% Verbessert den Textsatz durch z.B. geringfügige Buchstaben-Skalierungen

\renewcommand{\familydefault}{\sfdefault}
    
\usepackage{scrhack}% löst Probleme mit floats 
\usepackage[pdftex]{graphicx}

\usepackage{float}

\usepackage{pdflscape}

\usepackage{tournamentstyle}

\usepackage{pgfplotstable}

\begin{document}
    \pagenumbering{gobble}
    
    % print grouping
    \begin{landscape}
    	\centering
        \Huge{\textbf{\tournamentname}} \\
        \huge{\textbf{Groups}} \\
        \vspace{2em}
        \renewcommand{\arraystretch}{1.25}
        
        \LARGE
        \setlength{\tabcolsep}{1.5em} % Space between columns
	    \pgfplotstableset{
	       	col sep=comma, % Separator used in the CSV file
	       	string type, % Treat all columns as strings
	       	header=true, % Use the first row as headers
	       	assign column name/.style={/pgfplots/table/column name={\textbf{#1}}}, % Column names in bold
	    }
		\pgfplotstabletypeset{groups.csv}
    \end{landscape}
\end{document}