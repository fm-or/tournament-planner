\documentclass[
	headinclude=false,
	footinclude=false,
	]{scrartcl}
    
\usepackage[a4paper,margin=20mm]{geometry}

\usepackage[T1]{fontenc}% moderne Schriftkodierung
\usepackage[utf8]{inputenc}% Sonderzeichen vieler verschiedener Sprachen gleichzeitig (€, ° usw.)
\usepackage[english,main=ngerman]{babel}% Primär Deutsch und sekundär Englisch
\usepackage[autostyle]{csquotes}% Anführungszeichen mit \enquote{} passen zur Sprache von babel
\usepackage{lmodern}% Schriftfamilie latin modern basierend auf computer modern verträgt sich besser mit fontenc T1
\usepackage{microtype}% Verbessert den Textsatz durch z.B. geringfügige Buchstaben-Skalierungen

\renewcommand{\familydefault}{\sfdefault}
    
\usepackage{scrhack}% löst Probleme mit floats 
\usepackage[pdftex]{graphicx}

\usepackage{float}

\usepackage{ifthen}

\usepackage{tikz}

\usepackage{pdflscape}

\usepackage{tournamentstyle}

\usepackage{datatool}% load data from csv files
\usepackage{xstring} % for string operations (optional but convenient)

\begin{document}
    \pagenumbering{gobble}
    
    % print grouping
    \begin{landscape}
    	\centering
        \Huge{\textbf{\tournamentname}} \\
        \huge{\textbf{Groups}} \\
        \vspace{2em}
        \renewcommand{\arraystretch}{1.25}
        
        % Load the group data from csv
        \DTLloaddb{groupdata}{groups.csv}

        % Extract unique groups into a temporary macro for iteration
        \let\groupNames\empty

        % Iterate over all rows to collect unique groups
        \DTLforeach{groupdata}{\group=Group}{%
        	% Check if the group has already been found
        	\IfSubStr{\groupNames}{\group}{%
        	}{%
        		% Not found, add to list
        		\ifx\groupNames\empty%
        			% First entry, no leading comma
        			\edef\groupNames{\group}%
        		\else
        			% Not first, so repeat the already found group names and insert a comma
        			\edef\groupNames{\groupNames,\group}%
        		\fi
        	}%
        }

        % Now iterate over the unique groups and print tabulars
        \LARGE
        \setlength{\tabcolsep}{1em}
        \foreach \thisgroup in \groupNames {%
        	% Find the teams from this group and store them separated by \\
        	\let\teamNames\empty%
        	\DTLforeach{groupdata}{\group=Group,\team=Team}{%
        		\ifthenelse{\equal{\group}{\thisgroup}}{
       				\edef\teamNames{\teamNames\team\\}%
        		}{}%
        	}
        	% Print the teams inside a tabular
	        \begin{tabular}{|l|}
		        \hline
		        \textbf{\thisgroup}\\
		        \hline
		        \teamNames
		        \hline
		        \multicolumn{1}{c}{}\\
	        \end{tabular}%
	        \hspace{2em}
        }
        \hspace{-2.5em}
    \end{landscape}
\end{document}
