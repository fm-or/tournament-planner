\documentclass[
	a4paper,
	headinclude=false,
	footinclude=false,
	]{scrartcl}
    
\usepackage[margin=20mm,top=25mm]{geometry}

\usepackage[T1]{fontenc}% moderne Schriftkodierung
\usepackage[utf8]{inputenc}% Sonderzeichen vieler verschiedener Sprachen gleichzeitig (€, ° usw.)
\usepackage[english,main=ngerman]{babel}% Primär Deutsch und sekundär Englisch
\usepackage[autostyle]{csquotes}% Anführungszeichen mit \enquote{} passen zur Sprache von babel
\usepackage{lmodern}% Schriftfamilie latin modern basierend auf computer modern verträgt sich besser mit fontenc T1
\usepackage{microtype}% Verbessert den Textsatz durch z.B. geringfügige Buchstaben-Skalierungen

\renewcommand{\familydefault}{\sfdefault}
    
\usepackage{scrhack}% löst Probleme mit floats
\usepackage[pdftex]{graphicx}

\usepackage{tikz}
\usetikzlibrary{positioning,calc}

\usepackage{tournamentstyle}

\usepackage{datatool}% load data from csv files

\usepackage{xstring}% \IfSubStr

\newcommand{\createPoint}[3]{% Position, Präfix, Punkt
  \node[#1,draw,minimum height=8.5mm,minimum width=8.5mm] (#2-P#3) {#3};
}
\newcommand{\createPoints}[4]{% Position, Präfix, x, y
    \coordinate[#1] (#2-P0);
    \xdef\o{0}
    \xdef\q{0}
    \xdef\p{1}
    \foreach \y in {1, ..., #4}{
        \createPoint{below right=0mm of #2-P\o.south west}{#2}{\p}
        \xdef\o{\p}
        \pgfmathparse{int(\q+1)}
        \xdef\q{\pgfmathresult}
        \pgfmathparse{int(\p+1)}
        \xdef\p{\pgfmathresult}
        \foreach \x in {2, ..., #3}{
            \createPoint{right=0mm of #2-P\q}{#2}{\p}
            \pgfmathparse{int(\q+1)}
            \xdef\q{\pgfmathresult}
            \pgfmathparse{int(\p+1)}
            \xdef\p{\pgfmathresult}
        }
    }   
}

\usepackage{ulem}
\usepackage{amsmath}
%\usepackage{tabularx}
%\newcolumntype{R}[1]{>{\raggedleft\arraybackslash}p{#1}}
%\newcolumntype{L}[1]{>{\raggedright\arraybackslash}p{#1}}

\usepackage{array}
\newcolumntype{L}[1]{>{\raggedright\let\newline\\\arraybackslash\hspace{0pt}}m{#1}}
\newcolumntype{C}[1]{>{\centering\let\newline\\\arraybackslash\hspace{0pt}}m{#1}}
\newcolumntype{R}[1]{>{\raggedleft\let\newline\\\arraybackslash\hspace{0pt}}m{#1}}

\begin{document}
    \pagenumbering{gobble}
    \DTLloaddb{scheduledata}{schedule.csv}% load the schedule data from csv
    % find the court ids and store them
    \let\uniqueCourts\empty
    \DTLforeach{scheduledata}{\match=Match Nr,\court=Court,\serving=Team 1,\receiving=Team 2,\referee=Referee,\time=Time}{
    	\ifx\uniqueCourts\empty
    		% if no court has been stored yet, store the current court
    		\edef\uniqueCourts{\court}
    	\else
    		% otherwise and if the current court was not stored yet, append it with a comma
    		\IfSubStr{\uniqueCourts}{\court}{}{\edef\uniqueCourts{\uniqueCourts,\court}}
    	\fi
    }
    % print a match report page for each match sorted by court
    % first loop over the unique courts
    \foreach \currentCourt in \uniqueCourts {
    	% then loop over all matches
	    \DTLforeach{scheduledata}{\match=Match Nr,\court=Court,\serving=Team 1,\receiving=Team 2,\referee=Referee}{
	    	% print the page only if the match court equals the current court (because of sorting)
	    	\ifthenelse{\equal{\currentCourt}{\court}}{
	    		% start the match report on a new page
		        \newpage
		        % fill in information about this match (tournament name, match id, teams, referee, court)
		        \begin{center}
		            \huge{\textbf{\tournamentname}}\\
		            \vspace{1.25em}
		            \setlength{\ULdepth}{3pt}
		            \LARGE{\textbf{Match report: \uline{$\text{~\match~}$}}}\\
		            \vspace{1.5em}
		            \Large{
		                \renewcommand{\arraystretch}{0.8}
		                \begin{tabular}{@{}R{0.394\textwidth}@{}C{.25em}@{}C{.075\textwidth}@{}C{.25em}@{}L{.394\textwidth}@{}}
		                     \textbf{\serving} && \textbf{vs.} && \textbf{\receiving} \\[-.1em]
		                     \cline{1-2}\cline{4-5}
		                     \small{serve}
		                \end{tabular}
		            }\\[1.5em]
		            \large{
		                \renewcommand{\arraystretch}{0.7}
		                \begin{tabular}{@{}L{.128\textwidth}@{}C{.25em}@{}L{.394\textwidth}@{}C{.25em}@{}L{.347\textwidth}@{}}%.296
		                     \textbf{Referee:} && \referee && \\
		                     \cline{2-3}
		                \end{tabular}\\[1.25em]
		                \begin{tabular}{@{}L{.128\textwidth}@{}C{.25em}@{}C{.044\textwidth}@{}C{.25em}@{}L{.697\textwidth}@{}}%.296
			                \textbf{Court:} && \court && \\
			                \cline{2-3}
		                \end{tabular}
		            }
		        \end{center}
		        \vspace{-.5em}
		        % draw the template for the results (score, winner)
		        \begin{figure}[h]
		            \centering
		            \begin{tikzpicture}
		                \coordinate (origin) at (0, 0);
		                \createPoints{below right=0mm of origin}{T1}{5}{6}
		                \createPoints{above right=0mm and 20mm of T1-P5}{T2}{5}{6}
		    
		                \node[left=6mm of T1-P1] (points) {\Large \textbf{Score:}};
		                \node[right=6mm of T2-P5] {\phantom{\Large \textbf{Score:}}};
		    
		                \coordinate[below right=15mm and 0mm of T1-P26.south west] (result);
		                \draw[-] (result) -- (result -| T1-P30.south east);
		                \draw[-] (result -| T2-P26.south west) -- (result -| T2-P30.south east);
		                \node[anchor=193] at (result -| points.193) {\Large \textbf{Score:}};
		    
		                \coordinate (midTop) at ($(T1-P5.north)!0.5!(T2-P1.north)$);
		                \coordinate (midBot) at ($(T1-P30.south)!0.5!(T2-P26.south)$);
		                \draw[-] (midTop) + (0, .5) -- (result -| midBot) --  + (0, -.05);
		                
		                \coordinate[below=15mm of result] (winner);
		                \draw[-] (winner) -- (winner -| T2-P30.south east);
		                \node[anchor=191] at (winner -| points.191) {\Large \textbf{Winner:}};
		                
		                \coordinate[below=15mm of winner] (description);
		    
		                \node[anchor=west,align=left,text width=35em] at (description -| points.west) {In the event of a tie, please play a deciding point.\\Please bring this report directly to the tournament organizers after the end of the game!};
		            \end{tikzpicture}
		        \end{figure}
		    }{}
	    }%
	}
\end{document}
