\documentclass[
	headinclude=false,
	footinclude=false,
	]{scrartcl}
    
\usepackage[a3paper,margin=20mm]{geometry}

\usepackage[T1]{fontenc}% moderne Schriftkodierung
\usepackage[utf8]{inputenc}% Sonderzeichen vieler verschiedener Sprachen gleichzeitig (€, ° usw.)
\usepackage[english,main=ngerman]{babel}% Primär Deutsch und sekundär Englisch
\usepackage[autostyle]{csquotes}% Anführungszeichen mit \enquote{} passen zur Sprache von babel
\usepackage{lmodern}% Schriftfamilie latin modern basierend auf computer modern verträgt sich besser mit fontenc T1
\usepackage{microtype}% Verbessert den Textsatz durch z.B. geringfügige Buchstaben-Skalierungen

\renewcommand{\familydefault}{\sfdefault}
    
\usepackage{scrhack}% löst Probleme mit floats 
\usepackage[pdftex]{graphicx}

\usepackage{float}

\usepackage{ifthen}
\def\a{0} % Schalter zwischen Schablone/Zeitplan (0) und Ergebnissen (1)

\usepackage{tikz}
\usetikzlibrary{positioning}

\usepackage{tournamentstyle}

\usepackage{datatool}% load data from csv files

\def\b{5.5cm} % width of box
\def\c{5cm} % text width
\def\d{7cm} % text width
\tikzset{titlenode/.style={draw, minimum width=\b, minimum height=0.8cm, align=left}}
\tikzset{itemnode/.style={titlenode, text width=\c}}
\tikzset{itemnodeWide/.style={titlenode, text width=\d}}
\tikzset{smallnode/.style={draw, minimum height=0.8cm, minimum width= 1.5cm}}

\newcommand{\createGame}[8]{%
  \node[smallnode, #1] (G#2) {\strut #2};
  \node[itemnode, right=0mm of G#2] (G#2serve) {\strut #3};
  \node[itemnode,  right=0mm of G#2serve] (G#2receive) {\strut #4};
  \node[smallnode, right=0mm of G#2receive] (G#2court) {\strut Court #5};
  \node[itemnode, right=0mm of G#2court] (G#2referee) {\strut #6};
  \node[smallnode, text width=16mm, align =center, right=0mm of G#2referee] (G#2info) {\strut \ifthenelse{\a=1}{#8}{#7}};
}

\begin{document}
    \pagenumbering{gobble}
    
    % print schedule
    \centering
    \Huge{\textbf{\tournamentname}} \\
    \huge{\textbf{\ifthenelse{\a=1}{Procedure and Results}{Schedule and Template}}}
    \vspace{2em}
    
    \begin{figure}[H]
    	\centering
        \begin{tikzpicture}
        	\def\smallgap{5mm}
        	\def\mediumgap{10mm}
        	\def\largegap{15mm}
        	
            \coordinate (origin) at (0,0);
            
            \createGame{below=0mm of origin}{Nr}{Serving}{Receiving}{}{Referee}{Time}{Points}% this is the header

    		\DTLloaddb{scheduledata}{schedule.csv}% load the schedule data from csv
			\let\prevTime\empty% initialize the time of the previous row as empty
            \DTLforeach{scheduledata}{\match=Match Nr,\court=Court,\serving=Team 1,\receiving=Team 2,\referee=Referee,\time=Time}{% loop over the rows
				\ifx\prevTime\empty% if it is empty, we are still in the first row
					% the first row is mediumgap below the header
            		\expandafter\createGame
	               	\expandafter{below=\mediumgap of GNr}{\match}{\serving}{\receiving}{\court}{\referee}{\time}{:}
				\else
	               	\ifx\time\prevTime% if the time has not changed between rows, the games are in the same time slot so they are displayed without gap
	                	\expandafter\createGame
	                	\expandafter{below=0mm of G\number\numexpr\match-1\relax}{\match}{\serving}{\receiving}{\court}{\referee}{\time}{:}%
	               	\else% otherwise a new time slot is started by adding a smallgap between the rows
	                	\expandafter\createGame
	                	\expandafter{below=\smallgap of G\number\numexpr\match-1\relax}{\match}{\serving}{\receiving}{\court}{\referee}{\time}{:}%
	               	\fi
				\fi
	            \global\let\prevTime\time% the time of the previous row is updated
            }
        \end{tikzpicture}
    
    \end{figure}
\end{document}
